%&latex
\documentclass{jcgsf}

\setcounter{year}{2008}

\usepackage{url}
\def\UrlFont{\it}


\title{On General Adaptive Sparse Principal Component Analysis}


\author{Chenlei Leng and Hansheng Wang \thanks{Chenlei Leng is Assistant Professor at Department of Statistics and Applied Probability, National University of Singapore, Singapore, ({stalc@nus.edu.sg}). Hansheng Wang is Associate Professor at Guanghua School of Management, Peking University, Beijing, P. R. China, 100871 ({hansheng@gsm.pku.edu.cn})}}

\frenchspacing
\raggedbottom


\begin{document}

\maketitle

\begin{abstract}
The method of {\it sparse principal component
analysis} (S-PCA) proposed by Zou et al. (2006)
is an attractive approach to obtain sparse loadings in principal component
analysis (PCA).  S-PCA was motivated by reformulating PCA as a least squares
problem so that a lasso penalty on the loading coefficients can be applied.
In this article, we propose new estimates to improve S-PCA on the following
two aspects. Firstly, we propose a method of
{\it simple adaptive sparse principal component analysis} (SAS-PCA), which uses
the adaptive lasso penalty
(Zou, 2006; Wang et al., 2007) instead of the lasso penalty in S-PCA.
Secondly, we replace the least squares objective function in
S-PCA by a general least squares objective function. This formulation allows
us to study many related sparse PCA estimators under one unified theoretical framework
and leads to the method of {\it general adaptive sparse principal component analysis} (GAS-PCA).
Compared with SAS-PCA, GAS-PCA enjoys much improved finite sample performance.
In addition, we show that when a BIC-type criterion is used for selecting the
tuning parameters, the resulting estimates are consistent in variable selection. Numerical studies are conducted to
compare the finite sample performance of various competing methods.\\
\begin{keywords}
Adaptive Lasso; BIC; GAS-PCA; LARS; Lasso; S-PCA; SAS-PCA
\end{keywords}
\end{abstract}



%%%%%%%%%%%%%%%%%%%%%%%%%%%%%%%%%%%%%%%%%%%%%%%%%%%%%%%%%%%%%%%%

\bigskip

% AUTHOR: Please comment out any sections and/or files that you do not apply to your supplementary materials. You may also add more items, if you will will have more than two files of a particular type.

%AUTHOR: If you have a large number of files, please place them in a .tar or .zip file and comment out the note below.

%AUTHOR: Please rename this file using your manuscript number with the .tex extension. Then compile this file and mail (1) this .tex file, (2) the compiled .pdf file, and (3) your data files, computer code files, and supplementary documents to david.a.vandyk@gmail.com. Please put your manuscript number in the "Re" line of the e-mail. THANKS!

\centerline{\bf\large Supplementary Materials}

%The following files are all contained in the archive {\tt  archivename.tar or archivename.zip}.

\paragraph{1. Data Sets}
\begin{description}
\item[\tt teaching.txt] This is the dataset describing teaching evaluation scores of 251 courses in Guanghua School of Management, Peking University during the period from 2002 to 2004. Each row corresponds to one course and records the average scores. Each column corresponds to one of the following nine questions: 
\begin{itemize}
\item [(Q1)] I think this is a good course;
\item [(Q2)] The course improves my knowledge; 
\item [(Q3)] The schedule is reasonable; 
\item [(Q4)] The course is difficult;
\item [(Q5)] The course pace is too fast; 
\item [(Q6)] The course load is very heavy;
\item [(Q7)] The text book is good; 
\item [(Q8)] The reference book is helpful;
\item [(Q9)] Open this course is necessary.
\end{itemize}
\end{description}

\paragraph{2. Computer Code}
\begin{description}
\item[\tt GAS.r] The code used in the paper.
\item[\tt ex.r] A simple example on how to use GAS.r.
\end{description}



\end{document}
